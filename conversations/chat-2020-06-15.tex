\documentclass[a4paper, 14pt]{extarticle}
\usepackage{amsmath, amssymb, mathpazo}

\title{Polynomials and geometry}

\newcommand{\bC}{\mathbb{C}}

\begin{document}
\maketitle

\subsection*{Example constructions}

Line (by gradient and intercept): $y=mx+c$

Line (by two points):
\begin{align*}
  \frac{y-y_1}{y_2-y_1} &= \frac{x-x_1}{x_2-x_1}\\
  (y-y_1)(x_2-x_1) &= (x-x_1)(y_2-y_1)
\end{align*}

Circle: $(y-y_c)^2 + (x-x_c)^2 = r^2$

\subsection*{Polynomial algebra}

We work with a ring $\bC[x_1,\ldots,x_n]$, the ring of (multivariate)
polynomials in variables $x_1,\ldots,x_n$. If there are a small number
of variables, we'll call them $x, y, z$ rather than $x_1, x_2, x_3$.

A polynomial is a sum of \emph{terms}, each of which is a
\emph{coefficient} multiplied by a \emph{monomial}. So for example, if
I have
\[37x_1^2x_5 - x_2 + 10\]
then the term $37x_1^2x_5$ has the coefficient $37$ and the monomial
$x_1^2x_5$.

Instead of working with equations, we simply remember which things are
``supposed to be zero''. So instead of $y=mx+c$ we work with
$y-mx-c=0$.

An \emph{ideal} $I\subset\bC[x_1,\ldots,x_n]$ is a set of polynomials
with the following properties:
\begin{enumerate}
\item If $f,g\in I$, then $f+g\in I$;
\item If $f\in I$, and $g\in\bC[x_1,\ldots,x_n]$, then $fg\in I$.
\end{enumerate}

For a simple example of an ideal in $\bC[x,y]$, (if someone tells you
that $x=0$), the set of polynomials which are multiples of $x$ is an
ideal.  $x^3+xy\in I$, but $x^3 + xy - 7y \notin I$.

The general case looks like this: given elements
\[f_1,\ldots,f_d\in\bC[x_1,\ldots,x_n]\]
I can form an ideal
\[\left<f_1,\ldots,f_d\right> = \left\{f_1g_1+\cdots+f_dg_d\mid g_1,\ldots,g_d\in\bC[x_1,\ldots,x_n]\right\}\]

My simple example a moment ago was $\left<x\right>$.

The ideal $\left<f_1,\ldots,f_d\right>$ is the smallest ideal
containing $f_1,\ldots,f_d$.

In other words, if someone comes along and tells us that
$f_1,\ldots,f_d=0$, then $\left<f_1,\ldots,f_d\right>$ is the set of
polynomials we deduce are also zero by polynomial algebra.

So the basic problem is, now, given polynomials $f_1,\ldots,f_d,g$,
can we tell if $g\in\left<f_1,\ldots,f_d\right>$? It turns out that
this is a bit fiddly.

Consider $I = \left<x^{99}y^{100}+x, x^{100}y^{99}\right>$. This ideal contains
\[x(x^{99}y^{100}+x) - y(x^{100}y^{99}) = x^2.\]
As a result it also contains $x$, and in fact $I = \left<x\right>$.

Is there a means of calculating that makes it obvious that $x$ is in
this ideal? Yes, there is!

It's a two part plan:
\begin{enumerate}
\item start by transforming the ideal so that it has ``nice
  generators'',
\item persuade ourselves that, if it has ``nice generators'' then we
  can tell what's in it easily.
\end{enumerate}

The notion of ``nice generators'' is called being a \emph{Gr\"obner
  basis}.

\subsection*{To do\ldots}

Work out how to reproduce this in SAGE: work out to define a
polynomial ring $\bC[x,y]$, and define the ideal
$\left<x^{99}y^{100}+x, x^{100}y^{99}\right>$, and get SAGE to check
that a Gr\"obner basis for it is $\left<x\right>$.

Mess around a bit with other examples to get a feel.

Take a very small geometry problem, and see if you can completely
describe it as a question of polynomial algebra. Keep count of the
number of variables and equations. Can you get SAGE to do it with a
Grobner basis?

Proposed example: Thales's theorem (this that says that the angle in a
semicircle is a right angle).

\end{document}
