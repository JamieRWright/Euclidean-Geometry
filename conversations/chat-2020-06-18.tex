\documentclass[a4paper, 14pt]{extarticle}
\usepackage{amsmath, amssymb, mathpazo, tikz}

\title{More on ideals}

\begin{document}
\maketitle

\section{Ideal sums}

Let $R=k[x_1,\ldots,x_n]$, and let $I$ and $J$ be ideals of $R$.

We can define a new ideal
\[I+J=\left\{i+j\mid i\in I,j\in J\right\}.\]

If $I=\left<a_1,\ldots,a_r\right>$ and
$J=\left<b_1,\ldots,b_s\right>$, then
\[I+J=\left<a_1,\ldots,a_r,b_1,\ldots,b_s\right>.\]

(So ideal sums can be used to collect systems of equations together).

\section{``The two points such that\ldots''}

We've had a number of issues where things go wrong unless we assume
various points differ (for example, by assuming their $x$-coordinates
differ).

On the one hand, we should probably get into the habit of assuming our
inputs are nondegenerate (various points differ, triangles are not
collinear, etc).

In fact, the area of a triangle is given by the determinant
\begin{displaymath}
  \frac{1}{2}\left|\begin{matrix}
  1 & a_x & a_y\\
  1 & b_x & b_y\\
  1 & c_x & c_y
  \end{matrix}\right|,
\end{displaymath}
so assuming that that determinant has an inverse is assuming that the
triangle is nondegenerate.

There is another way, worth knowing of approaching questions like
this. Often, there are two points which work, and we want to say $A$
and $B$ are those two points, in some order. For example, we want to
be able to say ``let $A$ and $B$ be the two points where a line $\ell$
meets a circle $C$''.

In these cases, there's normally a quadratic equation floating around,
where the $x$-coordinates of $A$ and $B$ are the two roots.

This is true in our example: if we have line $\ell$ given by $y =
mx+c$, and we substitute that into the equation $(y-y_0)^2 + (x-x_0)^2
= r^2$ of a circle $C$.

There is a neat trick for saying $a$ and $b$ are the two roots of a
quadratic equation $uX^2 + vX + w$, namely we can use that, as
polynomials,
\[uX^2 + vX + w = u(X-a)(X-b)\]
which gives us equations
\begin{align*}
  u &= u\\
  v &= -u(a+b)\\
  w &= uab.
\end{align*}

\end{document}
