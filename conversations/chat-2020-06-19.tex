\documentclass[a4paper, 14pt]{extarticle}
\usepackage{amsmath, amssymb, mathpazo, tikz}

\title{Squares, square roots, etc}

\begin{document}
\maketitle

In the past we wanted to talk about $a$ and $b$ being equal, but in
fact only had access to $a^2$ and $b^2$.

Instead of working with the ideal $\left<a-b\right>$ we worked with
$\left<a^2-b^2\right>$.

That's quite harmless (mostly): if $a-b=0$, then $(a-b)(a+b)=0$, so
$a^2-b^2=0$. Alternatively, if $a-b=0$, then $a=b$, so $a^2=b^2$, so
$a^2-b^2=0$.


However, what if $a+b=c$, and we only have access to $a^2$, $b^2$ and
$c^2$? It just doesn't follow that $a^2+b^2=c^2$.

\end{document}
