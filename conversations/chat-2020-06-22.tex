\documentclass[a4paper, 14pt]{extarticle}
\usepackage{amsmath, amssymb, mathpazo, tikz}

\title{About Gr\"obner bases}

\newcommand{\bC}{\mathbb{C}}

\begin{document}
\maketitle

Consider $I = \left<x^{99}y^{100}+x, x^{100}y^{99}\right>$. This ideal contains
\[x(x^{99}y^{100}+x) - y(x^{100}y^{99}) = x^2.\]
As a result it also contains $x$, and in fact $I = \left<x\right>$.

We'll need to fix a monomial ordering. This means we can talk about
the \emph{leading term} of a polynomial.

The problem with that basis for $I$ is that $x\in I$, but it doesn't
look like it should be given the leading terms of our generators.

Here's a theorem (due to Buchberger). The following are equivalent for
an ideal $I$ and a basis $I=\left<f_1,\ldots,f_r\right>$:
\begin{enumerate}
\item If you take the ideal generated by all leading terms of elements
  of $I$, that's the same as the ideal generated by the leading terms
  of $f_1, \ldots, f_r$.
\item The leading term of any element of $I$ is divisible by the
  leading term of some element of $f_1,\ldots f_r$.
\item If you take any element of $\bC[x_1,\ldots,x_n]$ and repeatedly
  do polynomial division by $f_1,\ldots,f_r$, you get a unique remainder,
  no matter which order you do it in.
\end{enumerate}

For $I=\left<x^{99}y^{100}+x, x^{100}y^{99}\right>$, condition (1)
does not hold, because the leading terms of the generators are
$x^{99}y^{100}$ and $x^{100}y^{99}$, and those generate the ideal of
polynomials all of whose terms are divisible by $x^{99}y^{100}$ or
$x^{100}y^{99}$, but $x$ is a leading term of an element of $I$ but
not in that ideal.

Condition (2) also doesn't hold here because $x$ is not divisible by
$x^{99}y^{100}$ or $x^{100}y^{99}$.

Condition (3) also doesn't hold, because if I start with
$x^{100}y^{100}$, if I do division by $x^{99}y^{100}+x$ I get
remainder $-x^2$, whereas if I do division by $x^{100}y^{99}$ I get
remainder $0$.

However, you can check that $I=\left<x\right>$ is a good basis
satisfying all three of these conditions.

A basis satisfying these equivalent conditions is called a
\emph{Gr\"obner basis}.

\end{document}
